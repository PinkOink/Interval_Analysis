\include{settings}
\usepackage{amsmath}

\lstset{language=Java} 

\begin{document}	% начало документа

% Титульная страница
\begin{titlepage}	% начало титульной страницы

	\begin{center}		% выравнивание по центру

		\large Санкт-Петербургский политехнический университет Петра Великого\\
		\large Институт прикладной математики и механики \\
		\large Высшая школа прикладной математики и вычислительно физики \\[6cm]
		% название института, затем отступ 6см
		
		\huge Вычислительные комплексы\\[0.5cm] % название работы, затем отступ 0,5см
		\large \textbf{Лабораторная работа №3}\\[5.1cm]

	\end{center}


	\begin{flushright} % выравнивание по правому краю
		\begin{minipage}{0.25\textwidth} % врезка в половину ширины текста
			\begin{flushleft} % выровнять её содержимое по левому краю

				\large\textbf{Работу выполнил:}\\
				\large Колесник Виктор\\
				\large {Группа:} 3630102/70201\\
				
				\large \textbf{Преподаватель:}\\
				\large к.ф.-м.н., доцент\\
				\large Баженов Александр Николаевич

			\end{flushleft}
		\end{minipage}
	\end{flushright}
	
	\vfill % заполнить всё доступное ниже пространство

	\begin{center}
	\large Санкт-Петербург\\
	\large \the\year % вывести дату
	\end{center} % закончить выравнивание по центру

\end{titlepage} % конец титульной страницы

\vfill % заполнить всё доступное ниже пространство


% Содержание
\renewcommand\contentsname{\centerline{Содержание}}
\tableofcontents
\newpage

\listoffigures
\newpage


\section{Постановка задачи}
Необходимо решить прямоугольную систему уравнений субдифференциальным методом Ньютона путем нахождения решений с различными матрицами из исходной СЛАУ и взятием минимума по включению.


\section{Теория}
\subsection{Получение начального приближения}
Пусть задана ИСЛАУ:
\begin{equation}
	\textbf{C} x = \textbf{d}
\end{equation}
Для получения начального приближения нужно решить <<среднюю>> систему:
\begin{equation}
	\text{mid} \textbf{C}^\sim x^{(0)} = \text{sti} \textbf{d}
\end{equation}

\subsection{Субдифференциальный метод Ньютона}
Следующее приближение вычисляется по формуле:
\begin{equation}
	x^{(k)} = x^{(k - 1)} - \tau (D^{(k-1)})^{-1} \mathcal{G}(x^{k - 1})
\end{equation}
где $\tau \in (0,1]$ - некоторая константа. \\

Условием окончания алгоритма является выполнения условия:
\begin{equation}
	||x^{(k)} - x^{(k - 1)}|| \leq \varepsilon
\end{equation}

\subsection{Решение СЛАУ с прямоугольной матрицей}
Для прямоугольный матриц непосредственно применение субдифференциального метода Ньютона невозможно. Нужно сделать его частью более общего процесса, и в более общей постановке.
\begin{equation}
	\textbf{A x} \subseteq \textbf{b}
\end{equation}

Разобъем исходную матрицу $\textbf{A}$ на квадратные подматрицы $\textbf{A}^1$, $\textbf{A}^2$, ..., $\textbf{A}^k$. \\

Будем решать ИСЛАУ
\begin{equation}
	\textbf{A}^i \textbf{x} = \textbf{b}^i
\end{equation}
Затем решения $\textbf{x}^i$, $i=1, 2, ..., k$ пересекаем между собой:
\begin{equation}
	\textbf{x}^\star = \bigcap_{i=1}^{k} \text{pro} \textbf{x}^i
\end{equation}
Для персечения используем правильные проекции интервалов. Пересечение берется как минимум по включению в полной интервальной арифметике:
\begin{equation}
	\textbf{a} \bigcap \textbf{b} := \inf_{\subseteq} \{\textbf{a}, \textbf{b}\} = [\max \{ \overline{\textbf{a}}, \overline{\textbf{b}}\}, \min \{\underline{\textbf{a}}, \underline{\textbf{b}}\}]
\end{equation}


\section{Решение}
Дана ИСЛАУ, у которой матрица имеет размерность (126, 18). Правая часть является интервальным вектором со случайными радиусами из интервала $[0.1, 1]$. Элементы вектора-решения - случайные значения из интервала $[2, 8]$ \\

Будем случайно выбирать 18 строк из этой матрицы и решать соответствующую подсистему субдифференциальным метода Ньютона, если определитель матрицы не равен 0. \\

Затем, найдем пересечение полученных решений и сравним с истинным. \\

Проверим, как зависит получаемое решение от количества выборов подматриц. Для этого будем искать решение-пересечение при случайном выборе 1, 5, 15, 30, 50 или 100 подсистем и сравнивать с реальным. Кроме того, посмотрим, как получаемая правая часть соотносится с исходной для всей системы.



\section{Результаты}
Исходная прямоугольная матрица имеет следующий вид:

\begin{figure}[h]
	\centering
	\includegraphics[width=0.7\textwidth]{A_block_1.png}
	\caption{Исходная матрица}
\end{figure}

На следующих графиках представлены сравнения полученных правых частей и исходных и сравнения векторов-решений и истинных решений при выборе 1, 5, 15, 30, 50 или 100 подсистем.
\newpage
\begin{figure}[h]
	\centering
	\includegraphics[width=0.7\textwidth]{sols_1_mats_1.png}
	\caption{Сравнение правых частей. 1 подматрица}
\end{figure}

\begin{figure}[h]
	\centering
	\includegraphics[width=0.7\textwidth]{xs_1_mats_1.png}
	\caption{Сравнение исходного решения и полученного интервального решения. 1 подматрица}
\end{figure}
\newpage
\begin{figure}[h]
	\centering
	\includegraphics[width=0.7\textwidth]{sols_5_mats_1.png}
	\caption{Сравнение правых частей. 5 подматриц}
\end{figure}

\begin{figure}[h]
	\centering
	\includegraphics[width=0.7\textwidth]{xs_5_mats_1.png}
	\caption{Сравнение исходного решения и полученного интервального решения. 5 подматрица}
\end{figure}
\newpage
\begin{figure}[h]
	\centering
	\includegraphics[width=0.7\textwidth]{sols_15_mats_1.png}
	\caption{Сравнение правых частей. 15 подматриц}
\end{figure}

\begin{figure}[h]
	\centering
	\includegraphics[width=0.7\textwidth]{xs_15_mats_1.png}
	\caption{Сравнение исходного решения и полученного интервального решения. 15 подматриц}
\end{figure}
\newpage
\begin{figure}[h]
	\centering
	\includegraphics[width=0.7\textwidth]{sols_30_mats_1.png}
	\caption{Сравнение правых частей. 30 подматриц}
\end{figure}

\begin{figure}[h]
	\centering
	\includegraphics[width=0.7\textwidth]{xs_30_mats_1.png}
	\caption{Сравнение исходного решения и полученного интервального решения. 30 подматриц}
\end{figure}
\newpage
\begin{figure}[h]
	\centering
	\includegraphics[width=0.7\textwidth]{sols_50_mats_1.png}
	\caption{Сравнение правых частей. 50 подматриц}
\end{figure}

\begin{figure}[h]
	\centering
	\includegraphics[width=0.7\textwidth]{xs_50_mats_1.png}
	\caption{Сравнение исходного решения и полученного интервального решения. 50 подматриц}
\end{figure}
\newpage
\begin{figure}[h]
	\centering
	\includegraphics[width=0.7\textwidth]{sols_100_mats_1.png}
	\caption{Сравнение правых частей. 100 подматриц}
\end{figure}

\begin{figure}[h]
	\centering
	\includegraphics[width=0.7\textwidth]{xs_100_mats_1.png}
	\caption{Сравнение исходного решения и полученного интервального решения. 100 подматриц}
\end{figure}
\newpage

Проверим зависимость нормы разности исходного решения и модельного от количества используемых подматриц. Будем искать нормы разности вектора-решения и вектора правых границ модельных решений, вектора-решения и вектора левых границ модельных решений, вектора-решения и вектора середины интервалов модельных решений. Результаты представлены на графиках:
\begin{figure}[h]
	\centering
	\includegraphics[width=0.7\textwidth]{xinf_xsup_1.png}
	\caption{Сравнение норм разности исходного вектора-решения и границ интервалов полученных решений}
\end{figure}

\begin{figure}[h]
	\centering
	\includegraphics[width=0.7\textwidth]{xmid_1.png}
	\caption{Сравнение норм разности исходного вектора-решения и вектора-середины интервалов полученных решений}
\end{figure}
\newpage


\section{Анализ}
Из графиков видно, что для всех вариантов получаемая правая часть находилась в границах исходной правой части. Более того, при увеличении количества выбираемых подматриц интервалы правой части сужались. \\

Для всех вариантов истинный вектор-решение всегда находился в интервале полученного решения. А при увеличении количества подматриц полученный вектор сужался к истинному. \\

Середина полученного интервального вектора-решения почти сразу совпала с исходным решением. Норма разности имела порядок $10^{-14}$ для любого количества используемых подматриц. \\

Из этого можно сделать вывод, что при увеличении количества выбираемых матриц решение-пересечение стремится к истинному.



\section{Приложение}
Код программы на Python лежит в данном репозитории: \\
\url{https://github.com/PinkOink/Interval_Analysis/tree/main/course_project}{}


\end{document}